\documentclass[../main.tex]{subfiles}
\graphicspath{{\subfix{../images/}}}
\begin{document}

\section{Rutherfordův rozptyl}


Alfa částice (7,7 MeV) dopadají na hliníkovou nebo zlatou fólii. Jak nejblíže se může přiblížit?


Řešíme jako 1d srážku. Alfa částice přiletí a působí na ní Coloumbovská síla častice. Částice se obrací když se 
její kinetická energie rovná potenciální energie pole. 

V SI je energie 
\begin{equation}
    1.122×10^-12 J
\end{equation}





\subsubsection{Vztah}

\begin{equation}
    \frac{1}{4\pi \varepsilon_0} = 9*10^9
\end{equation}

Tenhle vztah plyne z toho že $\frac{1}{\varepsilon_0 \mu_0} = c^2$ a $\mu \approx 4\pi * 10^{-7}$


\subsection{Rozdělení co znát}

Maxwell Boltzmannovo


Fermi Dirac 

\subsection{Fermiho Diracovo rozdělení}

Fermiho energie - nejvyšší obsazený stav za 0vé energie. 


















\end{document}