\documentclass[../main.tex]{subfiles}
\graphicspath{{\subfix{../images/}}}
\begin{document}


Lorentzova síla
\begin{equation}
    \vec{F} = q(\vec{E} + \vec{v} \times \vec{B})
\end{equation}

\subsection{Částice co proletá spektroskopem}
Působí na ni elektrické pole. A zatím žádne magnetické.

\begin{equation}
    F_y = ma_y = qE 
\end{equation}

Electron je pak odkloněn o 

\begin{equation}
    \frac{v_y}{v_x} = \tan \theta = \frac{v_y}{v_x} = \frac{a_y t}{v_0} = \frac{q E}{m} \frac{l}{v_0^2}
\end{equation}

Relativní hmotnost lze pak spočítat jako 
\begin{equation}
    \frac{q }{m} = \frac{v_0^2}{E l} \tan \theta 
\end{equation}

Počítáme příklad 3.2.

Elektrony jsou urychleny 1000V.

Práce která na ně působí je 
\begin{equation}
    W = \int_{0}^{l} \vec{F}\cdot d\vec{x} = \int_{0}^{l} e \frac{U}{l} dx = e U 
\end{equation}
To v našem případě vyjde jako 1000eV.

\begin{remark}
    To je malé oproti klidové energii elektronu, to je $m_e c^2 = 511 keV$. 
    Protony ji mají větší asi o 1800. 
    Nejde tedy o relativistický jev.
\end{remark}

Jejich rychlost pak získáváme z 
\begin{equation}
    W = E_k = \frac{1}{2} m v^2
\end{equation}

Počítáme pro vodík.
\begin{equation}
    v_H = \sqrt{\frac{2W}{m_H}}
\end{equation}

Parametry máme jako 
\begin{equation}
    l = 5 cm
\end{equation}

\begin{equation}
    B = 0,05 T
\end{equation}

Navíc jednoduše víme 
\begin{equation}
    l = v_H t_H \implies t_H = \frac{l}{v_H}
\end{equation}

Síla působící na částici je
\begin{equation}
    F_H = e (v_H B)
\end{equation}

Newton 
\begin{equation}
    m_H \frac{dv}{dt} = F_h
\end{equation}

\begin{equation}
    v_t = \frac{F_H}{m_H} t_H = \frac{F_H}{m_H} \frac{l}{v_H}
\end{equation}

Částice je odkloněná o 
\begin{equation}
    \tan \theta = \frac{v_t}{v_H} = \frac{F_H}{m_H} \frac{l}{v_H^2} = \frac{ e l B}{m_v v_H} =  \frac{e l B}{m_H \sqrt{\frac{2 W }{m_H}}}
\end{equation}


Celkem dostaneme 
\begin{equation}
    \tan \theta = \frac{e l B}{\sqrt{2 m_H W}} \approx 0.55
\end{equation}

Na desce vzdálené 25 cm pak dostaneme
\begin{equation}
    y = \tan \theta * 25 cm \approx 14 cm
\end{equation}
Což je správný výsledek. 













\end{document}