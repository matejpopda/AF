\documentclass[../main.tex]{subfiles}
\graphicspath{{\subfix{../images/}}}
\begin{document}



\subsubsection{Výpočet Avogadrovy konstanty za pomoci rentgenové difrakce}

Uvažujeme krystal NaCl, víme hustotu
\begin{equation}
    \varrho [kg/m^3] = 2,16 g/cm^3 
\end{equation}

Dále víme Avogadrovo číslo
\begin{equation}
    N_a = 6,022 \times 10^{23} mol^{-1}
\end{equation}


\centeredimagenocaption{images/cubiclatticewithplanes.png}


Translační vektor - o kolik se musím posunout aby se periodicky opakovala struktura

Pro atomickou mřížku - základní elementární buňka je 2x2x2 krychle kde je Na a Cl na vrcholech.

Poté je efektivní plošný příspěvek každé kuličky 1/8. Je jich celkem 8.

Z toho získáváme molární hmotnost
\begin{equation}
    M_{m, NaCl} = 23 (\text{z Na}) + 35,45 (\text{z Cl}) = 58,5 [g/mol]
\end{equation}


Platí vztah
\begin{equation}
    \frac{\varrho}{M_m} N_a [cm^{-3}] = \frac{1}{V}  \approx 2.22 \times 10^{22} cm^{-3}
\end{equation}

(Tohle odvození co jsme dělali na hodině je špatně, podle mě by hmota v jedné buňce být polloviční)

Což je objem jedné elementární buňky. 

Z toho dostaneme že
\begin{equation}
    a = V^{\frac{1}{3}} = 3.5 \times10^{-10} m
\end{equation}

(Tohle by teda vyšlo trochu jinak)



\end{document}